
%提出するレポートの書式はこのtemplateファイルに沿って作成してください。
%特に表紙・概要の書式は変えないで下さい。

\documentclass[a4j]{jarticle}

\usepackage[dvipdfmx]{graphicx}
\usepackage{epsbox}
\usepackage{url}
\usepackage{here}
\usepackage{amsmath,amssymb}
\setlength{\headsep}{-5mm}
\setlength{\oddsidemargin}{0mm}
\setlength{\textwidth}{165mm}
\setlength{\textheight}{230mm}
\setlength{\footskip}{20mm}

\title{
\vspace{30mm}
{\bf 高知工科大学様} 
\\
\vspace{5mm}
大学掲示板(KUTBBS)\\
\vspace{5mm}
{\bf  外部設計書v0.1}
\vspace{90mm}
}

\author{
\vspace{5mm}
グループ10 \\
\vspace{5mm}
Pathfinder \\
\vspace{5mm}
\vspace{10mm}
}

%date{
%平成30年8月1日
%}

\begin{document}
\maketitle
\tableofcontents
\newpage




\section{システムの利用と業務の流れ}
KUTBBSは、学生が陥りやすいトラブルや、学生自身の学習における課題や悩みを自主的に解決するための掲示板型のウェブアプリケーションである。近年の情報収集におけるツールと言えば、専らスマートフォンを用いた情報検索であるが、それが必ずしも問題を解決するとは限らない。そこで、(本校の)学生同士が問題をネット上でかつ匿名に解決できるようにするために開発するシステムが、KUTBBSである。



本システムを利用する対象は、本校の学生である。また管理者は、本校の事務員を想定している。新規利用者は、まず事前に配布される仮ID・パスワードを受け取りこれを本システムに入力することで、ログインすることができる。仮ID・パスワードをそのまま使い続けるのはセキュリティ上好ましくないため、初回ログイン後速やかに変更を促す。管理者の登録については親管理者のみが行える仕様になっている(詳細な仕様はサブシステムの項目で述べる)。登録された管理者は、管理するための業務を行えるようになる。


登録を終えた利用者は、本システムを利用することが可能になる。利用者は、以下のような主な掲示板の機能を利用することができる。
\begin{itemize}
  \item スレッドの設立及び閲覧
  \item スレッドへの書き込み
  \item ブックマーク機能
  \item スレッド検索
  \item 良いレスに対してアクションを返す機能
  \item 不適切なレスやスレッドの通報
\end{itemize}






管理者が行う主な業務としては、仮ID・パスワード発行と、前述したような例にあたる不適切なスレッドやレスの削除である。また、不適切なスレッドやレスを繰り返し行う悪質なユーザに対して警告や、懲罰措置を行うことも想定されている。そのため、管理者は以下のような機能を利用することができる。
なお親管理者については、これらの操作に加えて管理者の登録も可能である。

\begin{itemize}
  \item 仮ID・及びパスワードの発行
  \item スレッドやレスの削除
  \item 通報されてきたスレッドやレスの閲覧
  \item ユーザの情報閲覧
  \item 不適切なユーザへの警告・懲罰機能
  \item 管理者の登録(親管理者のみ)
\end{itemize}





以下に本システムの利用の一例とその過程を示しながら、管理者の業務の流れを示す。

\begin{enumerate}
  \item 利用者Aは自身が興味のあるスレッドを閲覧し、書き込みを行った。スレッドを閲覧するために必要な情報は、対象のスレッドに書き込まれたレスの情報のみである。書き込みの際に必要となる情報は、自身の書き込み内容である。
		
  \item  利用者が書き込みを行った後、再度書き込みを行ったスレッドを閲覧すると、自分の書き込み宛に誹謗中傷をする書き込みがあったため、管理者に通報を行った。通報を行う際に必要な情報は、不適切なレスが行われたスレッドタイトルと書き込み内容、それに通報の理由である。
  \item  管理者は、Webブラウザから管理者ページにログインし、利用者から通報されてきたレス及びスレッドを確認するページを閲覧する。管理者は、そのページを閲覧し、公序良俗に著しく欠けるタイトルやレスに対して、削除を行った。
\end{enumerate}

図1は上記の流れを表したものである。


\begin{figure}[h!]
\begin{center}
\resizebox{10cm}{!}{\includegraphics{flow.png}}
\caption{上記の業務の流れ}
\label{fig:figuretest}
\end{center}
\end{figure}














\section{システム概要}


\subsection{掲示板機能}
\subsection{学生登録}
\subsection{管理者登録}



\section{利用者側のサブシステム設計}

\subsection{アカウント登録}

\subsection{ログインについて}


\subsection{書き込み}

\subsection{読み込み}

\subsection{スレッド立て}


\subsection{スレッド検索}

\subsubsection{カテゴリーについて}


\subsection{ブックマーク機能}

\subsection{通知機能}

\subsection{イイネ機能}

\subsection{拡張機能}

\subsection{マイページ}

\subsection{通報機能(?)}


\section{管理者側のサブシステム設計}


\subsection{アカウント登録}


\subsection{スレッド及びレスの削除機能}


\subsection{利用者へのお知らせ機能}


\subsection{レス自動修正}


\subsubsection{自動修正登録機能}


\subsection{ユーザ情報閲覧機能}


\subsection{悪質なユーザのBAN}


\subsection{仮ID・パスワード発行機能}



\section{ユーザインタフェース設計}

\subsection{画面遷移図}


\section{データベース設計}


\section{ネットワーク設計}





\bibliographystyle{jplain}
\begin{thebibliography}{10}



\end{thebibliography}


\end{document}

