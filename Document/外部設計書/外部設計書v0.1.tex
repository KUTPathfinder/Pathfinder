%提出するレポートの書式はこのtemplateファイルに沿って作成してください。
%特に表紙・概要の書式は変えないで下さい。

\documentclass[a4j]{jarticle}

\usepackage[dvipdfmx]{graphicx}
%\usepackage{epsbox}
\usepackage{url}
\usepackage{here}
\usepackage{amsmath,amssymb}
\setlength{\headsep}{-5mm}
\setlength{\oddsidemargin}{0mm}
\setlength{\textwidth}{165mm}
\setlength{\textheight}{230mm}
\setlength{\footskip}{20mm}

\title{
\vspace{30mm}
{\bf 高知工科大学様}
\\
\vspace{5mm}
大学掲示板(KUTBBS)\\
\vspace{5mm}
{\bf  外部設計書v0.1}
\vspace{90mm}
}

\author{
\vspace{5mm}
グループ10 \\
\vspace{5mm}
Pathfinder \\
\vspace{5mm}
\vspace{10mm}
}

%date{
%平成30年8月1日
%}

\begin{document}
\maketitle
\tableofcontents
\newpage




\section{システムの利用と業務の流れ}
本システムは、学生が陥りやすいトラブルや、学生自身の学習における課題や悩みを自主的に解決するための掲示板型のウェブアプリケーションである。近年の情報収集におけるツールと言えば、専らスマートフォンを用いた情報検索であるが、それが必ずしも問題を解決するとは限らない。そこで、(本校の)学生同士が問題をネット上でかつ匿名に解決できるようにするために開発するシステムが、本システムである。



本システムを利用する対象は、本校の学生である。また管理者は、本校の事務員を想定している。新規利用者は、まず事前に配布される仮ID・パスワードを受け取りこれを本システムに入力することで、ログインすることができる。仮ID・パスワードをそのまま使い続けるのはセキュリティ上好ましくないため、初回ログイン後速やかに変更を促す。管理者の登録については親管理者のみが行える仕様になっている(詳細な仕様はサブシステムの項目で述べる)。登録された管理者は、管理するための業務を行えるようになる。


登録を終えた利用者は、本システムを利用することが可能になる。利用者は、以下のような主な掲示板の機能を利用することができる。
\begin{itemize}
  \item スレッドの作成及び閲覧
  \item スレッドへの書き込み
  \item ブックマーク機能
  \item スレッド検索
  \item 良いレスに対してアクションを返す機能
  \item 不適切なレスやスレッドの通報
\end{itemize}



管理者が行う主な業務としては、仮ID・パスワード発行と、前述したような例にあたる不適切なスレッドやレスの削除である。また、不適切なスレッドやレスを繰り返し行う悪質なユーザに対して警告や、懲罰措置を行うことも想定されている。そのため、管理者は以下のような機能を利用することができる。
なお親管理者については、これらの操作に加えて管理者の登録も可能である。

\begin{itemize}
  \item 仮ID・及びパスワードの発行
  \item スレッドやレスの削除
  \item 通報されてきたスレッドやレスの閲覧
  \item ユーザの情報閲覧
  \item 不適切なユーザへの警告・懲罰機能
  \item 管理者の登録(親管理者のみ)
\end{itemize}





以下に本システムの利用の一例とその過程を示しながら、管理者の業務の流れを示す。

\begin{enumerate}
  \item 利用者Aは自身が興味のあるスレッドを閲覧し、書き込みを行った。スレッドを閲覧するために必要な情報は、対象のスレッドに書き込まれたレスの情報のみである。書き込みの際に必要となる情報は、自身の書き込み内容である。

  \item  利用者が書き込みを行った後、再度書き込みを行ったスレッドを閲覧すると、自分の書き込み宛に誹謗中傷をする書き込みがあったため、管理者に通報を行った。通報を行う際に必要な情報は、不適切なレスが行われたスレッドタイトルと書き込み内容、それに通報の理由である。
  \item  管理者は、Webブラウザから管理者ページにログインし、利用者から通報されてきたレス及びスレッドを確認するページを閲覧する。管理者は、そのページを閲覧し、公序良俗に著しく欠けるタイトルやレスに対して、削除を行った。
\end{enumerate}

図1は上記の流れを表したものである。


\begin{figure}[h!]
\begin{center}
\resizebox{10cm}{!}{\includegraphics{flow.png}}
\caption{上記の業務の流れ}
\label{fig:figuretest}
\end{center}
\end{figure}














\section{システム概要}
本システムは機能として下記に示す「ユーザ機能」「掲示板機能」、「利用者登録機能」、「管理者登録」の4つを実装する。
 また、各機能を構成するサブシステムも併せて示す。それぞれのサブシステムに関しては4節にて詳細を述べる。

\subsection{ユーザ機能}
\\ユーザ機能では、各ユーザが個人で書き込みに関する仕様を変更したり、ブックマークしたスレッドを容易に閲覧することができる。
\\【構築サブシステム】
\\ブックマーク閲覧サブシステム、マイページサブシステム

\subsection{掲示板機能}
\\掲示板機能では、大学の在学生同士の情報共有の場を提供する。
\\【構築サブシステム】
\\ ブックマーク登録サブシステム、お知らせサブシステム、検索サブシステム
\\ スレッドサブシステム、レスサブシステム、通報サブシステム、拡張機能サブシステム、通知サブシステム

\subsection{利用者登録機能}
\\利用者登録機能では、大学の新入生または転入生が本システムを利用できるようにアカウントを作成し、学生を掲示板に登録する。
\\【構築サブシステム】
\\ 登録情報変更サブシステム、登録情報通知サブシステム

\subsection{管理者登録}
\\管理者登録では、親管理者が掲示板の管理を行う子管理者を作成する。
\\【構築サブシステム】
\\ 子管理者管理サブシステム、お知らせ、編集サブシステム、掲示板編集サブシステム
\\ 利用者管理サブシステム、利用者登録サブシステム、管理者ログインサブシステム

\section{利用者側のサブシステム設計}
\subsection{サブシステム概要(利用者ページ)}
本システムのサブシステムには、下記のサブシステムが存在する。
\begin{itemize}
\item アカウント登録サブシステム\\
大学から与えられた初期IDとパスワードから自分が覚えやすいIDとパスワードへの変更によるアカウント登録を行う。

\item ログインサブシステム\\
設定されたIDとパスワードを入力することで自分のアカウントへのログインを行う。

\item 検索サブシステム\\
過去に作成されたスレッドやその中におけるレスの検索を行うことができる。

\item 通報サブシステム\\
閲覧者によって不愉快と感じるものについて通報することで管理者に通知させることができる。

\item ブックマークサブシステム\\
頻繁に訪問するスレッドを登録しておく機能。

\item 通知サブシステム\\
利用者が作成したスレッドに対して、レスが書き込まれた場合、または利用者が書き込んだレスに対して返信が書き込まれた場合に、利用者に通知を送ることができる。

\item コレクトボタンサブシステム\\
利用者が投稿したレスに対して、投稿した本人以外の利用者がその投稿内容が確からしいと判断したときに、コレクトボタンを押す。コレクトボタンを押された回数が多ければ多いほど、そのレスは信頼性が高いと判断できる。

\item マイページサブシステム\\
利用登録ページ(投稿者の表示名や画像を設定するページ)機能。
\end{itemize}

\subsection{サブシステム詳細(利用者ページ)}
各サブシステムの詳細を以下に示す。
\begin{itemize}
\item アカウント登録サブシステム\\
アカウント登録サブシステムは以下の手順を行うことで構成される。
\begin{enumerate}
\item 大学は各ユーザに対して仮IDと仮パスワードを発行する。
\item ユーザは与えられた仮IDと仮パスワードを用いて初回ログインを行う。
\item 初回ログイン時はIDとパスワードの変更画面に飛ぶようになっており、各ユーザが覚えやすいIDとパスワードへの変更を行う。
\item 変更後は次回以降のログインはその新しく設定したIDとパスワードを用いる。
\end{enumerate}

\item ログインサブシステム\\
ログインサブシステムは以下の機能によって構成される。
\begin{itemize}
\item 初回ログインのみ大学に与えられた仮IDと仮パスワードによりログインすることができる。
\item 二回目以降のログインは各ユーザは設定変更したIDとパスワードによりログインすることができる。
\end{itemize}

\item 検索サブシステム\\
検索サブシステムは以下の機能によって構成される。
\begin{itemize}
\item 検索できる内容
\begin{itemize}
\item 検索時点でのスレッド
\item 検索時点でのレス
\end{itemize}
\item 検索方法
\begin{itemize}
\item キーワード検索(and,or)
\item カテゴリ検索
\end{itemize}
\end{itemize}

\item 通報サブシステム\\
通報サブシステムは以下の機能によって構成される。

\begin{itemize}
\item 他のユーザによって作成されたスレッド自体に対して不適切なスレッドだと感じたら、通報することで管理者に通知させることができる。
\item スレッド内におけるメッセージに対しても同様に通報することで管理者に通知させることができる。\\
\end{itemize}
通報された内容が適切なものなら管理者から該当ユーザに注意喚起のメールを送信される。

\item ブックマークサブシステム\\
ブックマークサブシステムは以下の機能によって構成される。
\begin{itemize}
\item スレッドもしくはスレッド内のレスからユーザのブックマークを行うことができる。
\item ブックマークを行っているユーザを確認はマイページのブックマークリストから確認できる。
\item ブックマークリストから表示されているユーザの投稿一覧ページを閲覧できる。
\item ブックマークの解除を行うこともできる。
\end{itemize}

\item 通知サブシステム\\
通知サブシステムは以下の機能によって構成される。
\begin{itemize}
\item 各ユーザはスレッドに対して指定することができる。
\item 指定することでそのスレッドに新たな書き込みあがあれば、その直後にメッセージが表示しされ随時通知されるシステム。
\end{itemize}

\item いいねサブシステム\\
通知サブシステムは以下の機能によって構成される。
\begin{itemize}
\item スレッド内の本文もしくはレスに対していいねボタンを押すことができる。
\item コメントを入力しなくても意思表示を行うことができる。
\item いいねされた数は他のユーザに表示され、どのユーザがいいねしたのかを閲覧することができる。
\end{itemize}

\item マイページサブシステム\\
マイページサブシステムは以下の機能によって構成される。
\begin{itemize}
\item 初めて投稿をおこなう前に設定するようにメッセージが表示される。
\item メッセージにしたがって表示名やプロフィール等の設定を行う。
\end{itemize}
\end{itemize}



% \section{利用者側のサブシステム設計}
% \subsection{アカウント登録}
% \subsection{ログインについて}
% \subsection{書き込み}
% \subsection{読み込み}
% \subsection{スレッド立て}
% \subsection{スレッド検索}
% \subsubsection{カテゴリーについて}
% \subsection{ブックマーク機能}
% \subsection{通知機能}
% \subsection{イイネ機能}
% \subsection{拡張機能}
% \subsection{マイページ}
% \subsection{通報機能(?)}


\section{管理者側のサブシステム設計}

\subsection{サブシステム概要(管理者ページ)}
 管理者ページのサブシステムには下記のサブシステムが存在する。
\begin{enumerate}
  \item 子管理者管理サブシステム\\
  親管理者がIDとパスワードを発行することで、本システムを管理する子管理者を登録する。また、不要になった子管理者のアカウントを抹消する。
\\

  \item お知らせ表示サブシステム\\
  管理者から利用者全体に対して任意のお知らせを通達することができる。
\\

  \item 掲示板編集サブシステム\\
  管理者は不適切であると判断されたスレッドやレスを削除または置換することができる。
\\

  \item 利用者管理サブシステム\\
  管理者は不適切なスレッドやレスを書き込んだ利用者に対して警告やアカウント凍結などの処置を行うことができる。
\\

  \item 利用者登録サブシステム\\
  管理者は利用者の仮IDと仮パスワードの発行を行うことができる。仮IDと仮パスワードを用いて、利用者は本システムに初めてログインすることができる。
\\

  \item 管理者ログインサブシステム\\
  管理者は学内LANに接続された電子デバイスからのみ、管理者のIDとパスワードを入力することで、管理者として本システムにログインすることができる。
\end{enumerate}

\subsection{サブシステム詳細(管理者ページ)}
各サブシステムの詳細を以下に示す。
\begin{enumerate}

  \item 子管理者管理サブシステム\\
  子管理者管理サブシステムは以下の機能によって構成される。
  \begin{itemize}
    \item 子管理者用ID・パスワード発行機能\\
    親管理者は手動で子管理者用のIDとパスワードを登録し、発行することができる。発行用のページに遷移するには、親管理者のIDとパスワードが必要となる。
    \item 子管理者アカウント抹消機能\\
    親管理者は手動で不要になった子管理者のアカウントを抹消することができる。
    \item 子管理者用ID・パスワード変更機能\\
    親管理者は手動で子管理者のIDとパスワードを変更することができる。\\
  \end{itemize}


  \item お知らせ表示サブシステム\\
  お知らせ表示サブシステムは以下の機能によって構成される。
  \begin{itemize}
    \item お知らせ編集機能\\
    管理者は、利用者に対して通達するお知らせ内容を編集することができる。
    \item お知らせアップロード機能\\
    上記で編集したお知らせ内容を本システムトップページにアップロードすることができる。\\
  \end{itemize}


  \item 掲示板編集サブシステム\\
   掲示板編集サブシステムは以下の機能によって構成される。
  \begin{itemize}
    \item スレッド削除機能\\
    管理者は、データベースに存在するスレッド及びスレッドに格納された全てのレスを削除することができる。ただし、このときのスレッドは完全に削除されるわけではなく、削除履歴として別のテーブルに保存される。
    \item レス削除機能\\
    管理者は、データベースに存在するスレッド内のレスを削除することができる。ただし、このときのレスは完全に削除されるわけではなく、削除履歴として別のテーブルに保存される。
    \item 削除履歴閲覧機能\\
    管理者は、削除したスレッドまたはレスを一覧で閲覧することができる。
    \item レス置換機能\\
    管理者は、データベースに存在するスレッド内のレスの一部あるいは全文を別の文字列に手動で置換することができる。
    \item 不適切な単語登録機能\\
    管理者は、誹謗中傷や公序良俗に違反していると考えられる単語を登録することができる。
    \item 不適切な単語自動置換機能\\
    管理者が設定した不適切な単語を検出し次第、自動で伏せ字に置換することができる。
    \item 通報内容閲覧機能\\
    利用者によって通報されたスレッドまたはレスの内容の一覧を、通報理由と共に閲覧することができる。このとき、通報者のユーザ情報、通報されたスレッド主またはレスの書き込み主のユーザ情報も同時に表示される。\\
  \end{itemize}


  \item 利用者管理サブシステム\\
   利用者管理サブシステムは以下の機能によって構成される。
  \begin{itemize}
    \item 利用者情報閲覧機能\\
    管理者は、データベースに登録されている任意の利用者情報を閲覧することができる。
    \item 利用者に対する警告メール送信機能\\
    管理者は、不適切な内容のスレッド作成またはレスの書き込みを度々行った利用者に対して、注意喚起のメールを送信することができる。
    \item 利用者アカウント凍結機能\\
    管理者は、度重なる注意を通達したにも関わらず、迷惑行為が改善されない利用者のアカウントに対して、書き込み不可能にすることができる。\\
  \end{itemize}


  \item 利用者登録サブシステム\\
   利用者登録サブシステムは以下の機能によって構成される。
  \begin{itemize}
    \item 仮ID・仮パスワード発行機能\\
    管理者は仮IDと仮パスワードを発行することができる。仮IDと仮パスワードの生成には、乱数を用いる。\\
  \end{itemize}


  \item 管理者ログインサブシステム\\
   管理者ログインサブシステムは以下の機能によって構成される。
  \begin{itemize}
    \item 管理者認証機能\\
    本システムの管理作業を行う利用者が、管理者であるかの認証を、IDとパスワードを用いて行う。認証に成功した利用者は管理者として本システムの管理を行うことができる。
  \end{itemize}

\end{enumerate}



\section{ユーザインタフェース設計}

\subsection{画面遷移図}


\section{データベース設計}


\section{ネットワーク設計}





\bibliographystyle{jplain}
\begin{thebibliography}{10}



\end{thebibliography}


\end{document}