
%提出するレポートの書式はこのtemplateファイルに沿って作成してください。
%特に表紙・概要の書式は変えないで下さい。

\documentclass[a4j]{jarticle}

\usepackage[dvipdfmx]{graphicx}
\usepackage{epsbox}
\usepackage{url}
\usepackage{here}
\usepackage{amsmath,amssymb}
\setlength{\headsep}{-5mm}
\setlength{\oddsidemargin}{0mm}
\setlength{\textwidth}{165mm}
\setlength{\textheight}{230mm}
\setlength{\footskip}{20mm}

\title{
\vspace{30mm}
{\bf 高知工科大学様}
\\
\vspace{5mm}
大学掲示板(KUTBBS)\\
\vspace{5mm}
{\bf  外部設計書v0.1}
\vspace{90mm}
}

\author{
\vspace{5mm}
グループ10 \\
\vspace{5mm}
Pathfinder \\
\vspace{5mm}
\vspace{10mm}
}

%date{
%平成30年8月1日
%}

\begin{document}
\maketitle
\tableofcontents
\newpage




\section{システムの利用と業務の流れ}
KUTBBSは、学生が陥りやすいトラブルや、学生自身の学習における課題や悩みを自主的に解決するための掲示板型のウェブアプリケーションである。近年の情報収集におけるツールと言えば、専らスマートフォンを用いた情報検索であるが、それが必ずしも問題を解決するとは限らない。そこで、(本校の)学生同士が問題をネット上でかつ匿名に解決できるようにするために開発するシステムが、KUTBBSである。



本システムを利用する対象は、本校の学生である。また管理者は、本校の事務員を想定している。新規利用者は、まず事前に配布される仮ID・パスワードを受け取りこれを本システムに入力することで、ログインすることができる。仮ID・パスワードをそのまま使い続けるのはセキュリティ上好ましくないため、初回ログイン後速やかに変更を促す。管理者の登録については親管理者のみが行える仕様になっている(詳細な仕様はサブシステムの項目で述べる)。登録された管理者は、管理するための業務を行えるようになる。


登録を終えた利用者は、本システムを利用することが可能になる。利用者は、以下のような主な掲示板の機能を利用することができる。
\begin{itemize}
  \item スレッドの設立及び閲覧
  \item スレッドへの書き込み
  \item ブックマーク機能
  \item スレッド検索
  \item 良いレスに対してアクションを返す機能
  \item 不適切なレスやスレッドの通報
\end{itemize}



管理者が行う主な業務としては、仮ID・パスワード発行と、前述したような例にあたる不適切なスレッドやレスの削除である。また、不適切なスレッドやレスを繰り返し行う悪質なユーザに対して警告や、懲罰措置を行うことも想定されている。そのため、管理者は以下のような機能を利用することができる。
なお親管理者については、これらの操作に加えて管理者の登録も可能である。

\begin{itemize}
  \item 仮ID・及びパスワードの発行
  \item スレッドやレスの削除
  \item 通報されてきたスレッドやレスの閲覧
  \item ユーザの情報閲覧
  \item 不適切なユーザへの警告・懲罰機能
  \item 管理者の登録(親管理者のみ)
\end{itemize}





以下に本システムの利用の一例とその過程を示しながら、管理者の業務の流れを示す。

\begin{enumerate}
  \item 利用者Aは自身が興味のあるスレッドを閲覧し、書き込みを行った。スレッドを閲覧するために必要な情報は、対象のスレッドに書き込まれたレスの情報のみである。書き込みの際に必要となる情報は、自身の書き込み内容である。

  \item  利用者が書き込みを行った後、再度書き込みを行ったスレッドを閲覧すると、自分の書き込み宛に誹謗中傷をする書き込みがあったため、管理者に通報を行った。通報を行う際に必要な情報は、不適切なレスが行われたスレッドタイトルと書き込み内容、それに通報の理由である。
  \item  管理者は、Webブラウザから管理者ページにログインし、利用者から通報されてきたレス及びスレッドを確認するページを閲覧する。管理者は、そのページを閲覧し、公序良俗に著しく欠けるタイトルやレスに対して、削除を行った。
\end{enumerate}

図1は上記の流れを表したものである。


\begin{figure}[h!]
\begin{center}
\resizebox{10cm}{!}{\includegraphics{flow.png}}
\caption{上記の業務の流れ}
\label{fig:figuretest}
\end{center}
\end{figure}














\section{システム概要}
本システムは機能として下記に示す「掲示板機能」、「学生登録」、「管理者登録」の3つを実装する。
 また、各機能を構成するサブシステムも併せて示す。それぞれのサブシステムに関しては4節にて詳細を述べる。

\subsection{掲示板機能}
\\掲示板機能では、大学の在学生同士の情報共有の場を提供する。
\\【構築サブシステム】
\\ (サブシステム待ち)

\subsection{学生登録}
\\学生登録では、大学の新入生または転入生が本システムを利用できるようにアカウントを作成し、学生を掲示板に登録する。(要修正)
\\【構築サブシステム】
\\ (サブシステム待ち)

\subsection{管理者登録}
\\管理者登録では、親管理者が掲示板の管理を行う子管理者を作成する。
\\【構築サブシステム】
\\ (サブシステム待ち)

\section{サブシステム設計}
\subsection{サブシステム概要}
KUTBBSのサブシステムには、下記のサブシステムが存在する。
\begin{itemize}
%\item 掲示板機能   (これはサブではないよね??)
%機能選択
%スレッド作成・書き込み・閲覧(これらを詳細に)
%書き込み・読み込み
%スレッドへの新たな書き込み、もしくは過去に誰かが書き込みをしたスレッドの内容の読み込みを行える。
%\item スレッドの立ち上げ    (これも??)
%新たなスレッドの作成を自由に行うことができる。
\item アカウント登録サブシステム\\
大学から与えられた初期IDとパスワードから自分が覚えやすいIDとパスワードへの変更によるアカウント登録を行う。
\item ログインサブシステム\\
設定されたIDとパスワードを入力することで自分のアカウントへのログインを行う。
\item 検索サブシステム\\
過去に作成されたスレッドやその中におけるメッセージの検索を行うことができる。
\item 通報サブシステム\\
閲覧者によって不愉快と感じるものについて通報することで管理者に通知させることができる。
\item ブックマークサブシステム\\
頻繁に訪問するスレッドを登録しておく機能。
\item 通知サブシステム\\
掲示が新規に作成されたり、掲示が更新された時に、ユーザに更新情報を通知。
\item いいねサブシステム\\
本文やコメントに対して簡単に応答することができる。
\item マイページサブシステム\\
利用登録ページ(投稿者の表示名や画像を設定するページ)機能。
\end{itemize}

\subsection{サブシステム概要}
各サブシステムの詳細を以下に示す。
\begin{itemize}
\item アカウント登録サブシステム\\
ユーザは大学から与えられた仮IDと仮パスワードを用いて初回ログインを行う。この初回ログイン時にIDとパスワードの変更を行ってもらい、以後それらをもちいてログインをおこなってもらう。
\item ログインサブシステム\\
ユーザ自らが変更したIDとパスワードによりログインすることで掲示板への参加が可能となる。
\item 検索サブシステム\\
過去にどのようなスレッドがあり、どのような内容が書かれているいるのかの検索を行うことができる。検索方法としてはキーワード検索(and,or)とカテゴリ検索があり、スレッドもしくはメッセージ単体の検索を行うことができる。
\item 通報サブシステム\\
通報サブシステムは以下の機能によって構成される。
\begin{itemize}
\item 他のユーザによって作成されたスレッド自体に対して不適切なスレッドだと感じたら、通報することで管理者に通知させることができる。
\item スレッド内におけるメッセージに対しても同様に通報することで管理者に通知させることができる。\\
\end{itemize}
通報された内容が適切なものなら管理者から該当ユーザに注意喚起のメールを送信される。
\item ブックマークサブシステム\\
マイページからブックマークリストを確認でき、一覧が表示されるのでそこからユーザを確認し、その人の投稿一覧ページが表示される。
ブックマークから削除することも可能。
\item 通知サブシステム\\
ユーザは指定した掲示板のスレッドに書き込みがあると、その直後にメッセージが表示しされ随時通知されるシステム。
\item いいねサブシステム\\
ユーザが目にした本文やコメントに対して簡単に応答することができる。コメントを書き込まなくても意思表示っができるシステム。
\item マイページサブシステム\\
初めて投稿する時などに、表示されメッセージに従い表示名やプロフィール画像を設定する。後から変更することも可能。
\end{itemize}


\section{利用者側のサブシステム設計}
\subsection{アカウント登録}
\subsection{ログインについて}
\subsection{書き込み}
\subsection{読み込み}
\subsection{スレッド立て}
\subsection{スレッド検索}
\subsubsection{カテゴリーについて}
\subsection{ブックマーク機能}
\subsection{通知機能}
\subsection{イイネ機能}
\subsection{拡張機能}
\subsection{マイページ}
\subsection{通報機能(?)}


\section{管理者側のサブシステム設計}

\subsection{サブシステム概要(管理者ページ)}
 管理者ページのサブシステムには下記のサブシステムが存在する。
\begin{enumerate}
  \item 子管理者管理サブシステム\\
  親管理者がIDとパスワードを発行することで、KUTBBSを管理する子管理者を登録する。また、不要になった子管理者のアカウントを抹消する。
\\

  \item お知らせ表示サブシステム\\
  管理者から利用者に対して任意のお知らせを通達することができる。
\\

  \item 掲示板編集サブシステム\\
  管理者は不適切であると判断されたスレッドやメッセージを削除または置換することができる。
\\

  \item 利用者管理サブシステム\\
  管理者は不適切なスレッドやメッセージを書き込んだ利用者に対して警告やアカウント凍結などの処置を行うことができる。
\\

  \item 利用者登録サブシステム\\
  管理者は利用者の仮IDと仮パスワードの発行を行うことができる。これにより、利用者はKUTBBSに初めてログインすることができる。
\\

  \item 管理者ログインサブシステム\\
  管理者は学内LANに接続された電子デバイスからのみ、管理者のIDとパスワードを入力することで、管理者として本システムにログインすることができる。
\end{enumerate}

\subsection{サブシステム詳細(管理者ページ)}
各サブシステムの詳細を以下に示す。
\begin{enumerate}
  \item 子管理者管理サブシステム\\
  子管理者管理サブシステムは以下の機能によって構成される。
  \begin{itemize}
    \item 子管理者用ID・パスワード発行機能\\
    親管理者は手動で子管理者用のIDとパスワードを発行することができる。発行用のページに遷移するには、親管理者のIDとパスワードが必要となる。
    \item 子管理者アカウント抹消機能\\
    親管理者は手動で不要になった子管理者のアカウントを抹消することができる。
    \item 子管理者用ID・パスワード変更機能\\
    親管理者は手動で子管理者のIDとパスワードを変更することができる。\\
  \end{itemize}

  \item お知らせ表示サブシステム\\
  お知らせ表示サブシステムは以下の機能によって構成される。
  \begin{itemize}
    \item お知らせ編集機能\\
    管理者は、利用者に対して通達する内容を編集することができる。
    \item お知らせアップロード機能\\
    上記で編集したお知らせ内容をKUTBBSトップページにアップロードすることができる。\\
  \end{itemize}

  \item 掲示板編集サブシステム\\
   掲示板編集サブシステムは以下の機能によって構成される。
  \begin{itemize}
    \item スレッド削除機能\\
    管理者は、データベースに存在するスレッド及びスレッドに格納された全てのメッセージを削除することができる。ただし、このときのスレッドは完全に削除されるわけではなく、削除履歴として別のテーブルに保存される。
    \item メッセージ削除機能\\
    管理者は、データベースに存在するスレッド内のメッセージを削除することができる。ただし、このときのメッセージは完全に削除されるわけではなく、削除履歴として別のテーブルに保存される。
    \item 削除履歴閲覧機能\\
    管理者は、削除したスレッドまたはメッセージを一覧で閲覧することができる。
    \item メッセージ置換機能\\
    管理者は、データベースに存在するスレッド内のメッセージの一部あるいは全文を別の文字列に置換することができる。
    \item NGワード登録機能\\
    管理者は、コンプライアンス違反していると考えられる単語を登録することができる。
    \item NGワード自動置換機能\\
    上記で設定したNGワードを検出し次第、自動で伏せ字に変換することができる。
    \item 通報内容閲覧機能\\
    利用者によって通報されたスレッドまたはメッセージ内容を閲覧することができる。\\
  \end{itemize}

  \item 利用者管理サブシステム\\
   利用者管理サブシステムは以下の機能によって構成される。
  \begin{itemize}
    \item 利用者情報閲覧機能\\
    管理者は、データベースに登録されている任意の利用者情報を閲覧することができる。
    \item 利用者に対する警告メール送信機能\\
    管理者は、不適切な内容のスレッド作成またはメッセージの書き込みを度々行った利用者に対して、注意喚起のメールを送信することができる。
    \item 利用者アカウント凍結機能\\
    管理者は、度重なる注意を通達したにも関わらず、迷惑行為が改善されない利用者のアカウントに対して、書き込み不可能にすることができる。\\
  \end{itemize}

  \item 利用者登録サブシステム\\
   利用者登録サブシステムは以下の機能によって構成される。
  \begin{itemize}
    \item 仮ID・仮パスワード発行機能\\
    管理者は仮IDと仮パスワードを発行することができる。仮IDと仮パスワードの生成には、乱数を用いる。\\
  \end{itemize}

  \item 管理者ログインサブシステム\\
   管理者ログインサブシステムは以下の機能によって構成される。
  \begin{itemize}
    \item 管理者認証機能\\
    KUTBBSの管理作業を行う利用者が、管理者であるかの認証を、IDとパスワードを用いて行う。認証に成功した利用者は管理者として本システムの管理を行うことができる。
  \end{itemize}

\end{enumerate}



\section{ユーザインタフェース設計}

\subsection{画面遷移図}


\section{データベース設計}


\section{ネットワーク設計}





\bibliographystyle{jplain}
\begin{thebibliography}{10}



\end{thebibliography}


\end{document}
